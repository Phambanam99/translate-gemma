% ============================================================================
% BÁO CÁO TÍNH NĂNG CHIẾN LƯỢC KỸ THUẬT
% Dự án: CSV Translator Pro - TranslateGemma AI
% Mục đích: Trình hội đồng phê duyệt kế hoạch thực hiện
% ============================================================================

\documentclass[a4paper,12pt]{article}

% ======================== PACKAGES ========================
\usepackage[utf8]{inputenc}
\usepackage[vietnamese]{babel}
\usepackage[T5]{fontenc}
\usepackage{geometry}
\usepackage{graphicx}
\usepackage{booktabs}
\usepackage{longtable}
\usepackage{array}
\usepackage{multirow}
\usepackage{xcolor}
\usepackage{colortbl}
\usepackage{hyperref}
\usepackage{listings}
\usepackage{fancyhdr}
\usepackage{titlesec}
\usepackage{enumitem}
\usepackage{amsmath}
\usepackage{tabularx}
\usepackage{float}
\usepackage{caption}
\usepackage{tcolorbox}
\usepackage{tikz}
\usetikzlibrary{shapes.geometric, arrows.meta, positioning, calc, decorations.pathreplacing}

% ======================== PAGE SETUP ========================
\geometry{
    top=2.5cm,
    bottom=2.5cm,
    left=3cm,
    right=2.5cm,
}

% ======================== COLORS ========================
\definecolor{primaryblue}{RGB}{25, 55, 109}
\definecolor{accentgreen}{RGB}{16, 185, 129}
\definecolor{warningorange}{RGB}{245, 158, 11}
\definecolor{dangerred}{RGB}{239, 68, 68}
\definecolor{lightgray}{RGB}{243, 244, 246}
\definecolor{codebg}{RGB}{248, 249, 250}
\definecolor{codegray}{rgb}{0.5,0.5,0.5}
\definecolor{codegreen}{rgb}{0,0.6,0}
\definecolor{codepurple}{rgb}{0.58,0,0.82}

% ======================== HEADER/FOOTER ========================
\pagestyle{fancy}
\fancyhf{}
\fancyhead[L]{\small\textcolor{primaryblue}{Báo cáo tính năng chiến lược kỹ thuật -- CSV Translator Pro}}
\fancyhead[R]{\small\textcolor{primaryblue}{\thepage}}
\fancyfoot[C]{\small\textcolor{codegray}{Phần mềm dịch offline AI TranslateGemma -- Trình hội đồng phê duyệt}}
\renewcommand{\headrulewidth}{0.5pt}
\renewcommand{\footrulewidth}{0.3pt}

% ======================== LISTING STYLE ========================
\lstset{
    backgroundcolor=\color{codebg},
    basicstyle=\ttfamily\small,
    breaklines=true,
    captionpos=b,
    commentstyle=\color{codegreen},
    keywordstyle=\color{primaryblue}\bfseries,
    stringstyle=\color{codepurple},
    numberstyle=\tiny\color{codegray},
    numbers=left,
    numbersep=8pt,
    frame=single,
    rulecolor=\color{lightgray},
    tabsize=4,
    showstringspaces=false,
}

% ======================== TCOLORBOX STYLES ========================
\tcbuselibrary{skins, breakable}

\newtcolorbox{infobox}[1][]{
    colback=blue!5!white,
    colframe=primaryblue,
    fonttitle=\bfseries,
    title=#1,
    breakable,
}

\newtcolorbox{warnbox}[1][]{
    colback=orange!5!white,
    colframe=warningorange,
    fonttitle=\bfseries,
    title=#1,
    breakable,
}

\newtcolorbox{successbox}[1][]{
    colback=green!5!white,
    colframe=accentgreen,
    fonttitle=\bfseries,
    title=#1,
    breakable,
}

\newtcolorbox{dangerbox}[1][]{
    colback=red!5!white,
    colframe=dangerred,
    fonttitle=\bfseries,
    title=#1,
    breakable,
}

% ======================== TITLE FORMATTING ========================
\titleformat{\section}
    {\Large\bfseries\color{primaryblue}}{\thesection}{1em}{}
\titleformat{\subsection}
    {\large\bfseries\color{primaryblue!80}}{\thesubsection}{1em}{}
\titleformat{\subsubsection}
    {\normalsize\bfseries\color{primaryblue!60}}{\thesubsubsection}{1em}{}

% ======================== HYPERLINK ========================
\hypersetup{
    colorlinks=true,
    linkcolor=primaryblue,
    urlcolor=primaryblue,
    citecolor=primaryblue,
}

% ============================================================================
\begin{document}

% ======================== TRANG BÌA ========================
\begin{titlepage}
    \centering
    \vspace*{1.5cm}
    
    {\huge\bfseries\textcolor{primaryblue}{BÁO CÁO TÍNH NĂNG}}\\[0.5cm]
    {\huge\bfseries\textcolor{primaryblue}{CHIẾN LƯỢC KỸ THUẬT}}\\[1cm]
    
    {\LARGE\textcolor{primaryblue}{Phần Mềm Dịch Thuật Offline}}\\[0.3cm]
    {\Large\textcolor{primaryblue}{CSV Translator Pro}}\\[1.5cm]
    
    \rule{\textwidth}{1.5pt}\\[0.5cm]
    {\Large\textbf{Sử dụng mô hình AI TranslateGemma của Google}}\\[0.3cm]
    {\large Hoạt động hoàn toàn offline -- Hỗ trợ 55 ngôn ngữ}\\[0.3cm]
    {\large Kiến trúc Client-Server hiện đại}\\[0.5cm]
    \rule{\textwidth}{1.5pt}\\[1.5cm]
    
    \begin{tabular}{rl}
        \textbf{Mục đích:} & Trình hội đồng phê duyệt kế hoạch thực hiện \\[0.3cm]
        \textbf{Quy mô triển khai:} & 1000 người dùng \\[0.3cm]
        \textbf{Mô hình AI:} & TranslateGemma (4B / 27B tham số) \\[0.3cm]
        \textbf{Công nghệ:} & FastAPI + React + PyTorch + CUDA \\[0.3cm]
        \textbf{Tổng ngân sách dự kiến:} & 789.800.000 VNĐ \\[0.3cm]
    \end{tabular}
    
    \vfill
    {\large Ngày lập: \today}
\end{titlepage}

% ======================== MỤC LỤC ========================
\tableofcontents
\newpage

% ============================================================================
%                     PHẦN 1: TÓM TẮT ĐIỀU HÀNH
% ============================================================================
\section{Tóm tắt điều hành (Executive Summary)}

\subsection{Bối cảnh và nhu cầu}

Trong bối cảnh hội nhập quốc tế, đơn vị thường xuyên phải xử lý khối lượng lớn văn bản đa ngôn ngữ, đặc biệt là \textbf{tiếng Ả Rập}. Việc dịch thủ công tốn nhiều thời gian và nguồn lực, trong khi các dịch vụ dịch thuật trực tuyến (Google Translate, DeepL) \textbf{không đáp ứng yêu cầu bảo mật} vì dữ liệu phải gửi qua Internet.

\subsection{Giải pháp đề xuất}

Triển khai hệ thống \textbf{CSV Translator Pro} --- phần mềm dịch thuật AI \textbf{hoàn toàn offline}, đáp ứng:

\begin{itemize}[leftmargin=1.5cm]
    \item \textbf{Bảo mật tuyệt đối}: Dữ liệu không rời khỏi mạng nội bộ
    \item \textbf{Dịch hàng loạt}: Xử lý file CSV chứa hàng nghìn dòng văn bản
    \item \textbf{Chất lượng cao}: Sử dụng mô hình AI tiên tiến của Google
    \item \textbf{Đa ngôn ngữ}: Hỗ trợ 55 ngôn ngữ, đặc biệt tốt với tiếng Ả Rập
\end{itemize}

\subsection{Điểm nhấn chiến lược}

\begin{table}[H]
\centering
\renewcommand{\arraystretch}{1.4}
\begin{tabularx}{\textwidth}{|l|X|}
\hline
\rowcolor{primaryblue!10}
\textbf{Tiêu chí} & \textbf{Giá trị} \\
\hline
Tổng ngân sách & \textbf{789.800.000 VNĐ} (bao gồm phần cứng + triển khai + dự phòng 10\%) \\
\hline
Thời gian triển khai & \textbf{6 tuần} (từ khảo sát đến nghiệm thu) \\
\hline
Số người dùng & \textbf{1000 người} (mở rộng được) \\
\hline
Mô hình AI & \textbf{TranslateGemma-27B} (27 tỷ tham số, chất lượng production) \\
\hline
GPU Server & \textbf{NVIDIA A100 80GB} (chạy full precision BF16) \\
\hline
Khả năng offline & \textbf{100\%} (sau khi tải model lần đầu) \\
\hline
\end{tabularx}
\end{table}

\begin{successbox}[Khuyến nghị của đội dự án]
Đề xuất hội đồng \textbf{phê duyệt kế hoạch} triển khai với cấu hình GPU NVIDIA A100 80GB để đảm bảo:
\begin{itemize}
    \item Chất lượng dịch thuật tối ưu (mô hình 27B full precision)
    \item Hiệu năng cao phục vụ 1000 người dùng đồng thời
    \item Khả năng mở rộng trong tương lai
    \item Độ tin cậy enterprise với bảo hành dài hạn
\end{itemize}
\end{successbox}

% ============================================================================
%                     PHẦN 2: TÍNH NĂNG CHÍNH
% ============================================================================
\newpage
\section{Tính năng chính của hệ thống}

\subsection{Tổng quan 3 chức năng dịch thuật}

\begin{figure}[H]
\centering
\begin{tikzpicture}[
    node distance=0.8cm,
    feature/.style={
        rectangle, rounded corners=8pt, draw=primaryblue, thick,
        fill=blue!8, text width=13cm, minimum height=1.8cm,
        font=\small
    },
]

\node[feature] (f1) {
    \textbf{1. DỊCH FILE CSV HÀNG LOẠT}\\[0.2cm]
    Upload file CSV $\to$ Chọn cột cần dịch $\to$ Chọn ngôn ngữ nguồn/đích $\to$ Theo dõi tiến trình $\to$ Tải file kết quả\\
    \textit{Ứng dụng: Dịch danh sách tàu thuyền, báo cáo tình báo, dữ liệu Thuraya...}
};

\node[feature, below=of f1] (f2) {
    \textbf{2. DỊCH VĂN BẢN TRỰC TIẾP}\\[0.2cm]
    Nhập văn bản $\to$ Chọn ngôn ngữ $\to$ Nhấn Dịch $\to$ Nhận kết quả ngay\\
    \textit{Ứng dụng: Dịch tin nhắn, email, đoạn văn ngắn cần xử lý nhanh}
};

\node[feature, below=of f2] (f3) {
    \textbf{3. DỊCH VĂN BẢN TỪ ẢNH (OCR)}\\[0.2cm]
    Upload ảnh $\to$ AI nhận dạng chữ $\to$ Dịch tự động $\to$ Hiển thị kết quả\\
    \textit{Ứng dụng: Dịch ảnh chụp tài liệu, biển báo, ảnh màn hình}
};

\end{tikzpicture}
\caption{Ba chức năng dịch thuật chính của CSV Translator Pro}
\end{figure}

\subsection{Chi tiết tính năng dịch file CSV}

Đây là tính năng \textbf{chiến lược} của hệ thống, cho phép xử lý hàng loạt dữ liệu:

\begin{table}[H]
\centering
\caption{Quy trình dịch file CSV}
\renewcommand{\arraystretch}{1.3}
\begin{tabularx}{\textwidth}{|c|X|X|}
\hline
\rowcolor{primaryblue!10}
\textbf{Bước} & \textbf{Thao tác} & \textbf{Chi tiết kỹ thuật} \\
\hline
1 & Upload file CSV & Hỗ trợ file lớn, mã hóa UTF-8, tự động detect header \\
\hline
2 & Chọn cột cần dịch & Mặc định cột ``Text'', có thể tùy chỉnh \\
\hline
3 & Chọn ngôn ngữ & 55 ngôn ngữ hỗ trợ (Ả Rập, Việt, Anh, Trung, Nhật...) \\
\hline
4 & Bắt đầu dịch & Server xử lý bất đồng bộ, không block giao diện \\
\hline
5 & Theo dõi tiến trình & Progress bar cập nhật mỗi giây, hiển thị \% và số dòng \\
\hline
6 & Tải kết quả & File CSV mới với cột ``Translated\_Text'' thêm vào \\
\hline
\end{tabularx}
\end{table}

\begin{infobox}[Hiệu năng ước tính với GPU A100 80GB]
\renewcommand{\arraystretch}{1.3}
\begin{tabular}{|l|c|c|}
\hline
\rowcolor{primaryblue!10}
\textbf{Số dòng CSV} & \textbf{Thời gian ước tính} & \textbf{Tokens/giây} \\
\hline
100 dòng & 2--3 phút & 40--70 \\
\hline
500 dòng & 8--12 phút & 40--70 \\
\hline
1,000 dòng & 15--22 phút & 40--70 \\
\hline
5,000 dòng & 1--2 giờ & 40--70 \\
\hline
\end{tabular}
\end{infobox}

\subsection{Danh sách 55 ngôn ngữ hỗ trợ}

Mô hình TranslateGemma hỗ trợ dịch qua lại giữa \textbf{55 ngôn ngữ}, bao gồm:

\begin{table}[H]
\centering
\caption{Các ngôn ngữ được cấu hình sẵn trong hệ thống}
\renewcommand{\arraystretch}{1.3}
\begin{tabular}{|c|l||c|l||c|l|}
\hline
\rowcolor{primaryblue!10}
\textbf{Mã} & \textbf{Ngôn ngữ} & \textbf{Mã} & \textbf{Ngôn ngữ} & \textbf{Mã} & \textbf{Ngôn ngữ} \\
\hline
\texttt{ar} & Tiếng Ả Rập & \texttt{vi} & Tiếng Việt & \texttt{en} & Tiếng Anh \\
\hline
\texttt{zh} & Tiếng Trung & \texttt{ja} & Tiếng Nhật & \texttt{ko} & Tiếng Hàn \\
\hline
\texttt{ru} & Tiếng Nga & \texttt{de} & Tiếng Đức & \texttt{fr} & Tiếng Pháp \\
\hline
\texttt{es} & Tiếng Tây Ban Nha & \texttt{pt} & Tiếng Bồ Đào Nha & \texttt{it} & Tiếng Ý \\
\hline
\texttt{nl} & Tiếng Hà Lan & \texttt{pl} & Tiếng Ba Lan & \texttt{tr} & Tiếng Thổ Nhĩ Kỳ \\
\hline
\texttt{th} & Tiếng Thái & \texttt{cs} & Tiếng Séc & \multicolumn{2}{c|}{\textit{... và 38 ngôn ngữ khác}} \\
\hline
\end{tabular}
\end{table}

% ============================================================================
%                     PHẦN 3: KIẾN TRÚC KỸ THUẬT
% ============================================================================
\newpage
\section{Kiến trúc kỹ thuật hệ thống}

\subsection{Kiến trúc tổng quan Client-Server}

\begin{figure}[H]
\centering
\begin{tikzpicture}[
    node distance=2cm,
    box/.style={
        rectangle, rounded corners=5pt, draw=primaryblue, thick,
        fill=blue!5, text width=3.2cm, minimum height=1.8cm,
        text centered, font=\small
    },
    bigbox/.style={
        rectangle, rounded corners=5pt, draw=primaryblue, thick,
        fill=blue!5, text width=3.5cm, minimum height=2cm,
        text centered, font=\small
    },
    arrow/.style={-{Stealth[length=3mm]}, thick, primaryblue}
]

% Clients
\node[box] (client1) {
    \textbf{CLIENTS}\\[0.1cm]
    1000 trình duyệt Web\\
    Chrome / Edge / Firefox
};

% Server
\node[bigbox, right=2.5cm of client1] (server) {
    \textbf{SERVER}\\[0.1cm]
    FastAPI + Uvicorn\\
    Python 3.12\\
    CORS enabled
};

% AI Model
\node[bigbox, right=2.5cm of server] (model) {
    \textbf{AI MODEL}\\[0.1cm]
    TranslateGemma-27B\\
    PyTorch 2.6 + CUDA\\
    Hugging Face
};

% GPU
\node[box, below=1.5cm of model] (gpu) {
    \textbf{GPU SERVER}\\[0.1cm]
    NVIDIA A100 80GB\\
    HBM2e 2TB/s
};

% LAN label
\node[draw=accentgreen, thick, rounded corners=4pt, fill=green!5, 
      minimum width=2cm, minimum height=0.6cm, font=\footnotesize,
      above=1cm of server] (lan) {
    \textbf{Mạng LAN 1--10 Gbps}
};

% Arrows
\draw[arrow] (client1) -- node[above, font=\footnotesize] {HTTP API} (server);
\draw[arrow] (server) -- node[above, font=\footnotesize] {Inference} (model);
\draw[arrow] (model) -- (gpu);

\end{tikzpicture}
\caption{Sơ đồ kiến trúc hệ thống CSV Translator Pro}
\end{figure}

\subsection{Stack công nghệ}

\begin{table}[H]
\centering
\caption{Công nghệ sử dụng trong hệ thống}
\renewcommand{\arraystretch}{1.4}
\begin{tabularx}{\textwidth}{|l|l|X|}
\hline
\rowcolor{primaryblue!10}
\textbf{Thành phần} & \textbf{Công nghệ} & \textbf{Mô tả} \\
\hline
\multicolumn{3}{|c|}{\cellcolor{accentgreen!15}\textbf{Backend (Server)}} \\
\hline
Framework web & FastAPI & Framework Python hiệu năng cao, async native \\
\hline
ASGI Server & Uvicorn & Xử lý requests đồng thời, production-ready \\
\hline
ML Framework & PyTorch 2.6 & Deep learning framework của Meta \\
\hline
Model Loading & Hugging Face Transformers & Quản lý và chạy mô hình AI \\
\hline
Quantization & bitsandbytes & Lượng tử hóa 4-bit/8-bit giảm VRAM \\
\hline
\multicolumn{3}{|c|}{\cellcolor{blue!10}\textbf{Frontend (Client)}} \\
\hline
UI Library & React 19.2 & Component-based UI framework \\
\hline
Build Tool & Vite 7 & Build tool siêu nhanh, Hot Module Replacement \\
\hline
\multicolumn{3}{|c|}{\cellcolor{warningorange!10}\textbf{AI Model}} \\
\hline
Mô hình chính & TranslateGemma-27B-IT & 27 tỷ tham số, Google research \\
\hline
CUDA & 12.4 & GPU acceleration cho NVIDIA \\
\hline
\end{tabularx}
\end{table}

\subsection{API Endpoints}

\begin{table}[H]
\centering
\caption{Danh sách API của hệ thống}
\renewcommand{\arraystretch}{1.3}
\begin{tabularx}{\textwidth}{|l|l|X|}
\hline
\rowcolor{primaryblue!10}
\textbf{Method} & \textbf{Endpoint} & \textbf{Chức năng} \\
\hline
GET & \texttt{/api/languages} & Lấy danh sách 55 ngôn ngữ hỗ trợ \\
\hline
POST & \texttt{/api/translate-text} & Dịch một đoạn văn bản \\
\hline
POST & \texttt{/api/translate-image} & OCR + dịch văn bản từ ảnh \\
\hline
POST & \texttt{/api/upload} & Upload CSV và bắt đầu dịch hàng loạt \\
\hline
GET & \texttt{/api/status/\{job\_id\}} & Kiểm tra tiến trình dịch CSV \\
\hline
GET & \texttt{/api/download/\{job\_id\}} & Tải file CSV đã dịch \\
\hline
GET & \texttt{/api/health} & Kiểm tra trạng thái hoạt động server \\
\hline
\end{tabularx}
\end{table}

% ============================================================================
%                     PHẦN 4: MÔ HÌNH AI TRANSLATEGEMMA
% ============================================================================
\newpage
\section{Mô hình AI TranslateGemma}

\subsection{Giới thiệu mô hình}

\textbf{TranslateGemma} là dòng mô hình dịch thuật mới nhất của Google, được xây dựng trên nền tảng kiến trúc \textbf{Gemma 3}. Đây là mô hình \textbf{mã nguồn mở} (Open Source) có thể chạy hoàn toàn offline sau khi tải về.

\begin{table}[H]
\centering
\caption{Thông số mô hình TranslateGemma-27B-IT}
\renewcommand{\arraystretch}{1.4}
\begin{tabularx}{\textwidth}{|l|X|}
\hline
\rowcolor{primaryblue!10}
\textbf{Thông số} & \textbf{Giá trị} \\
\hline
Tên mô hình & \texttt{google/translategemma-27b-it} \\
\hline
Số tham số & 27 tỷ (27B) \\
\hline
Kiến trúc & Decoder-only Transformer (Gemma 3) \\
\hline
Số lớp (Layers) & 46 \\
\hline
Hidden Dimension & 4096 \\
\hline
Attention Heads & 64 (Grouped-Query Attention) \\
\hline
Context Length & 2,048 tokens \\
\hline
Số ngôn ngữ hỗ trợ & 55 ngôn ngữ \\
\hline
Framework huấn luyện & JAX + ML Pathways (Google TPU) \\
\hline
Dữ liệu huấn luyện & 4.3 tỷ tokens (SFT) + 10.2 triệu tokens (RLHF) \\
\hline
Giấy phép & Gemma License (mã nguồn mở) \\
\hline
\end{tabularx}
\end{table}

\subsection{So sánh chất lượng dịch thuật}

\begin{table}[H]
\centering
\caption{Benchmark so sánh các phiên bản TranslateGemma}
\renewcommand{\arraystretch}{1.4}
\begin{tabular}{|l|c|c|c|c|}
\hline
\rowcolor{primaryblue!10}
\textbf{Benchmark} & \textbf{4B} & \textbf{12B} & \textbf{27B} & \textbf{Ghi chú} \\
\hline
WMT24++ MetricX $\downarrow$ & 5.32 & 3.60 & \textbf{3.09} & Thấp hơn = tốt hơn \\
\hline
WMT24++ COMET $\uparrow$ & 81.6 & 83.5 & \textbf{84.4} & Cao hơn = tốt hơn \\
\hline
Vistra MetricX $\downarrow$ & 2.57 & 2.08 & \textbf{1.57} & Đánh giá đa ngôn ngữ \\
\hline
\end{tabular}
\end{table}

\begin{successbox}[Lợi ích của mô hình 27B]
Mô hình TranslateGemma-27B cho \textbf{chất lượng dịch thuật vượt trội}:
\begin{itemize}
    \item Giảm 42\% lỗi dịch so với phiên bản 4B (MetricX 5.32 $\to$ 3.09)
    \item Xử lý tốt các ngữ cảnh phức tạp, thuật ngữ chuyên ngành
    \item Đặc biệt hiệu quả với tiếng Ả Rập (ngôn ngữ khó)
    \item Hỗ trợ OCR từ ảnh với độ chính xác cao
\end{itemize}
\end{successbox}

\subsection{Yêu cầu VRAM theo phương pháp lượng tử hóa}

\begin{table}[H]
\centering
\caption{Yêu cầu VRAM GPU cho mô hình 27B}
\renewcommand{\arraystretch}{1.4}
\begin{tabularx}{\textwidth}{|l|c|c|X|}
\hline
\rowcolor{primaryblue!10}
\textbf{Phương pháp} & \textbf{VRAM Model} & \textbf{VRAM Thực tế} & \textbf{Ghi chú} \\
\hline
FP16 / BF16 (Full) & $\sim$54 GB & $\sim$58--62 GB & Chất lượng tốt nhất, cần A100/H100 \\
\hline
INT8 (8-bit) & $\sim$27 GB & $\sim$30--34 GB & Giảm nhẹ chất lượng, cần A6000 48GB \\
\hline
NF4 (4-bit) & $\sim$14 GB & $\sim$18--22 GB & Giảm chất lượng, chạy được RTX 4090 \\
\hline
\end{tabularx}
\end{table}

% ============================================================================
%                     PHẦN 5: CẤU HÌNH PHẦN CỨNG
% ============================================================================
\newpage
\section{Cấu hình phần cứng đề xuất}

\subsection{Cấu hình Server với GPU A100 80GB}

Đây là cấu hình được \textbf{khuyến nghị} cho triển khai production với 1000 người dùng:

\begin{table}[H]
\centering
\caption{Chi tiết cấu hình Server GPU NVIDIA A100 80GB}
\renewcommand{\arraystretch}{1.4}
\begin{tabularx}{\textwidth}{|l|X|c|r|}
\hline
\rowcolor{primaryblue!10}
\textbf{Thành phần} & \textbf{Thông số kỹ thuật} & \textbf{SL} & \textbf{Thành tiền (VNĐ)} \\
\hline
GPU & NVIDIA A100 80GB PCIe (HBM2e, 2TB/s) & 1 & 450.000.000 \\
\hline
CPU & AMD EPYC 7313 (16C/32T, 3.0GHz) & 1 & 55.000.000 \\
\hline
Mainboard & Supermicro H12SSL-i (SP3 Socket) & 1 & 25.000.000 \\
\hline
RAM & DDR4-3200 ECC REG 128GB (8x16GB) & 1 bộ & 35.000.000 \\
\hline
SSD & NVMe Gen4 2TB (Samsung PM9A3 Enterprise) & 1 & 12.000.000 \\
\hline
PSU & 1200W 80+ Platinum Redundant & 1 & 8.000.000 \\
\hline
Case & 4U Rackmount Server Chassis & 1 & 10.000.000 \\
\hline
Cooling & Hệ thống tản nhiệt Server & 1 & 5.000.000 \\
\hline
UPS & 3000VA Online Double Conversion & 1 & 25.000.000 \\
\hline
\multicolumn{3}{|r|}{\textbf{Tổng chi phí phần cứng}} & \textbf{625.000.000} \\
\hline
\end{tabularx}
\end{table}

\begin{successbox}[Tại sao chọn NVIDIA A100 80GB?]
\begin{itemize}
    \item \textbf{VRAM 80GB HBM2e}: Chạy mô hình 27B với Full Precision BF16 --- không cần quantization, chất lượng dịch tối ưu
    \item \textbf{Tensor Cores thế hệ 3}: Tăng tốc inference AI lên 312 TFLOPS
    \item \textbf{Băng thông 2TB/s}: Xử lý batch lớn không bị bottleneck
    \item \textbf{ECC Memory}: Đảm bảo độ chính xác tính toán
    \item \textbf{Thiết kế Datacenter}: Hoạt động 24/7, bảo hành enterprise
\end{itemize}
\end{successbox}

\subsection{Yêu cầu máy khách (Client)}

Máy khách \textbf{không cần cấu hình cao} vì toàn bộ AI chạy trên Server:

\begin{table}[H]
\centering
\caption{Yêu cầu phần cứng/phần mềm máy khách}
\renewcommand{\arraystretch}{1.3}
\begin{tabularx}{\textwidth}{|l|X|X|}
\hline
\rowcolor{primaryblue!10}
\textbf{Thành phần} & \textbf{Tối thiểu} & \textbf{Khuyến nghị} \\
\hline
CPU & Intel Core i3 / AMD Ryzen 3 & Intel Core i5 trở lên \\
\hline
RAM & 4 GB & 8 GB \\
\hline
Màn hình & 1280$\times$720 & 1920$\times$1080 (Full HD) \\
\hline
Trình duyệt & Chrome / Edge / Firefox $\geq$ 90 & Chrome mới nhất \\
\hline
Mạng & Kết nối LAN/WiFi đến Server & LAN Gigabit \\
\hline
\end{tabularx}
\end{table}

% ============================================================================
%                     PHẦN 6: CHI PHÍ DỰ TOÁN
% ============================================================================
\newpage
\section{Chi phí dự toán tổng hợp}

\subsection{Chi phí phần cứng}

\begin{table}[H]
\centering
\caption{Dự toán chi phí phần cứng Server}
\renewcommand{\arraystretch}{1.3}
\begin{tabularx}{\textwidth}{|X|r|}
\hline
\rowcolor{primaryblue!10}
\textbf{Hạng mục} & \textbf{Chi phí (VNĐ)} \\
\hline
GPU NVIDIA A100 80GB PCIe & 450.000.000 \\
\hline
CPU AMD EPYC 7313 & 55.000.000 \\
\hline
Mainboard Supermicro H12SSL-i & 25.000.000 \\
\hline
RAM 128GB DDR4 ECC & 35.000.000 \\
\hline
SSD NVMe 2TB Enterprise & 12.000.000 \\
\hline
PSU 1200W Platinum + Case + Cooling & 23.000.000 \\
\hline
UPS 3000VA Online & 25.000.000 \\
\hline
\textbf{Tổng phần cứng} & \textbf{625.000.000} \\
\hline
\end{tabularx}
\end{table}

\subsection{Chi phí triển khai và đào tạo}

\begin{table}[H]
\centering
\caption{Dự toán chi phí triển khai}
\renewcommand{\arraystretch}{1.3}
\begin{tabularx}{\textwidth}{|X|r|}
\hline
\rowcolor{primaryblue!10}
\textbf{Hạng mục} & \textbf{Chi phí (VNĐ)} \\
\hline
Triển khai phần mềm (cài đặt, cấu hình, kiểm thử 2 tuần) & 30.000.000 \\
\hline
Đào tạo quản trị viên (2--3 người, 2 ngày) & 5.000.000 \\
\hline
Đào tạo người dùng (1000 người, 10 đợt) & 50.000.000 \\
\hline
Tài liệu hướng dẫn (biên soạn, in ấn) & 3.000.000 \\
\hline
Hỗ trợ kỹ thuật ban đầu (1 tháng sau nghiệm thu) & 5.000.000 \\
\hline
\textbf{Tổng triển khai \& đào tạo} & \textbf{93.000.000} \\
\hline
\end{tabularx}
\end{table}

\subsection{Tổng hợp ngân sách}

\begin{table}[H]
\centering
\caption{Tổng hợp ngân sách dự án}
\renewcommand{\arraystretch}{1.5}
\begin{tabular}{|l|r|}
\hline
\rowcolor{primaryblue!10}
\textbf{Hạng mục} & \textbf{Chi phí (VNĐ)} \\
\hline
Phần cứng Server (A100 80GB + hệ thống) & 625.000.000 \\
\hline
Triển khai \& Đào tạo & 93.000.000 \\
\hline
Dự phòng (10\%) & 71.800.000 \\
\hline
\rowcolor{accentgreen!20}
\textbf{TỔNG CỘNG} & \textbf{789.800.000} \\
\hline
\end{tabular}
\end{table}

\begin{infobox}[Phân tích chi phí theo đầu người sử dụng]
\begin{itemize}
    \item Tổng chi phí: \textbf{789.800.000 VNĐ} cho 1000 người sử dụng
    \item Chi phí trung bình: \textbf{789.800 VNĐ/người} (đầu tư 1 lần)
    \item Chi phí phần mềm: \textbf{Miễn phí} (mã nguồn mở Google)
    \item Chi phí vận hành: Chỉ điện năng + bảo trì thường xuyên
    \item So sánh: Dịch vụ dịch thuật trung bình \textbf{200.000 VNĐ/1000 từ}
\end{itemize}
\end{infobox}

% ============================================================================
%                     PHẦN 7: TIMELINE TRIỂN KHAI
% ============================================================================
\newpage
\section{Kế hoạch thời gian triển khai}

\subsection{Tổng quan 5 giai đoạn (6 tuần)}

\begin{figure}[H]
\centering
\begin{tikzpicture}[
    node distance=0.5cm,
    phase/.style={
        rectangle, rounded corners=5pt, draw=primaryblue, thick,
        fill=blue!8, text width=14cm, minimum height=1.4cm,
        font=\small
    },
    arrow/.style={-{Stealth[length=2mm]}, thick, primaryblue!60}
]

\node[phase] (p1) {\textbf{Giai đoạn 1: Chuẩn bị} (Tuần 1--2)\\
Khảo sát hạ tầng, đặt mua phần cứng, chuẩn bị tài liệu, nhận thiết bị};

\node[phase, below=of p1] (p2) {\textbf{Giai đoạn 2: Triển khai Server} (Tuần 3)\\
Lắp đặt, cài OS/Driver/CUDA, cài Python/PyTorch, tải mô hình AI, cấu hình mạng};

\node[phase, below=of p2] (p3) {\textbf{Giai đoạn 3: Đóng gói \& Kiểm thử} (Tuần 4)\\
Build Frontend, đóng gói offline, kiểm thử đầy đủ, sửa lỗi \& tối ưu};

\node[phase, below=of p3] (p4) {\textbf{Giai đoạn 4: Đào tạo} (Tuần 5)\\
Đào tạo quản trị viên, đào tạo 1000 người dùng (10 đợt), giải đáp thắc mắc};

\node[phase, below=of p4] (p5) {\textbf{Giai đoạn 5: Nghiệm thu \& Bàn giao} (Tuần 6)\\
Kiểm thử UAT, hoàn thiện tài liệu, họp nghiệm thu, ký biên bản, bàn giao};

\draw[arrow] (p1) -- (p2);
\draw[arrow] (p2) -- (p3);
\draw[arrow] (p3) -- (p4);
\draw[arrow] (p4) -- (p5);

\end{tikzpicture}
\caption{Sơ đồ 5 giai đoạn triển khai trong 6 tuần}
\end{figure}

\subsection{Nhân sự triển khai}

\begin{table}[H]
\centering
\caption{Đội ngũ triển khai dự án}
\renewcommand{\arraystretch}{1.3}
\begin{tabularx}{\textwidth}{|c|l|X|c|}
\hline
\rowcolor{primaryblue!10}
\textbf{STT} & \textbf{Vai trò} & \textbf{Trách nhiệm} & \textbf{Số lượng} \\
\hline
1 & Quản lý dự án (PM) & Điều phối, báo cáo tiến độ, giao tiếp các bên & 1 \\
\hline
2 & KTV Phần cứng & Lắp đặt Server, kiểm tra thiết bị & 1 \\
\hline
3 & KTV Hệ thống & Cài đặt OS, Driver, CUDA & 1 \\
\hline
4 & KTV Phần mềm & Cài đặt ứng dụng, cấu hình, debug & 1--2 \\
\hline
5 & KTV Mạng & Cấu hình mạng, firewall, IP & 1 \\
\hline
6 & QA/Tester & Kiểm thử chức năng, hiệu năng & 1 \\
\hline
7 & Đào tạo viên & Đào tạo người dùng cuối & 1--2 \\
\hline
\multicolumn{3}{|r|}{\textbf{Tổng cộng}} & \textbf{8--10 người} \\
\hline
\end{tabularx}
\end{table}

% ============================================================================
%                     PHẦN 8: RỦI RO VÀ GIẢM THIỂU
% ============================================================================
\newpage
\section{Đánh giá rủi ro và phương án giảm thiểu}

\subsection{Ma trận rủi ro}

\begin{table}[H]
\centering
\caption{Đánh giá các rủi ro chính}
\renewcommand{\arraystretch}{1.4}
\footnotesize
\begin{tabularx}{\textwidth}{|c|X|c|c|X|}
\hline
\rowcolor{primaryblue!10}
\textbf{ID} & \textbf{Rủi ro} & \textbf{Xác suất} & \textbf{Mức} & \textbf{Phương án giảm thiểu} \\
\hline
R1 & Chậm giao hàng GPU A100 & TB & \cellcolor{warningorange!30}Cao & Đặt hàng sớm, có 2--3 nhà cung cấp dự phòng \\
\hline
R2 & GPU lỗi hoặc không tương thích & Thấp & \cellcolor{yellow!30}TB & Kiểm tra kỹ trước mua, yêu cầu bảo hành \\
\hline
R3 & Mạng LAN không ổn định & Thấp & \cellcolor{green!20}Thấp & Khảo sát và nâng cấp mạng trước triển khai \\
\hline
R4 & Mất điện đột ngột & Thấp & \cellcolor{yellow!30}TB & UPS 3000VA + yêu cầu generator dự phòng \\
\hline
R5 & Người dùng khó tiếp cận & TB & \cellcolor{green!20}Thấp & Đào tạo kỹ, tài liệu đơn giản, video hướng dẫn \\
\hline
R6 & Bảo mật: Truy cập trái phép & Thấp & \cellcolor{yellow!30}TB & Firewall, chỉ mở trong LAN, không public \\
\hline
\end{tabularx}
\end{table}

\subsection{Phương án dự phòng GPU}

\begin{warnbox}[Nếu không mua được A100 80GB]
Trong trường hợp GPU A100 80GB không khả dụng hoặc vượt ngân sách:
\begin{itemize}
    \item \textbf{Phương án B}: RTX A6000 48GB + INT8 quantization (chi phí $\sim$300 triệu, chất lượng giảm 5--10\%)
    \item \textbf{Phương án C}: RTX 4090 24GB + NF4 4-bit quantization (chi phí $\sim$100 triệu, chất lượng giảm 15--20\%)
    \item \textbf{Phương án D}: Sử dụng mô hình TranslateGemma-4B thay vì 27B (chất lượng giảm 40\%, nhưng chạy được trên GPU 8GB)
\end{itemize}
\end{warnbox}

% ============================================================================
%                     PHẦN 9: BẢO MẬT VÀ TUÂN THỦ
% ============================================================================
\newpage
\section{Bảo mật và tuân thủ}

\subsection{Cam kết bảo mật dữ liệu}

\begin{table}[H]
\centering
\caption{Các biện pháp bảo mật hệ thống}
\renewcommand{\arraystretch}{1.4}
\begin{tabularx}{\textwidth}{|l|X|}
\hline
\rowcolor{primaryblue!10}
\textbf{Biện pháp} & \textbf{Chi tiết} \\
\hline
\textbf{Hoạt động offline} & Sau khi cài đặt, hệ thống hoạt động 100\% offline, không cần Internet \\
\hline
\textbf{Dữ liệu nội bộ} & Tất cả dữ liệu dịch thuật xử lý trong mạng LAN, không gửi ra ngoài \\
\hline
\textbf{Firewall} & Chỉ mở port 8000 trong mạng nội bộ, không public ra Internet \\
\hline
\textbf{Không lưu log dịch} & Hệ thống không lưu trữ nội dung văn bản đã dịch (chỉ log kỹ thuật) \\
\hline
\textbf{Xóa dữ liệu tạm} & File CSV upload được xóa tự động sau 7 ngày \\
\hline
\textbf{Không cần đăng nhập} & Truy cập trong LAN (có thể thêm authentication nếu cần) \\
\hline
\end{tabularx}
\end{table}

\begin{successbox}[Điểm mạnh về bảo mật]
\textbf{So với dịch vụ dịch thuật online (Google Translate, DeepL):}
\begin{itemize}
    \item Dữ liệu KHÔNG rời khỏi mạng đơn vị
    \item Không phụ thuộc Internet --- hoạt động trong mọi điều kiện
    \item Không lo ngại điều khoản sử dụng của bên thứ ba
    \item Toàn quyền kiểm soát hệ thống và dữ liệu
\end{itemize}
\end{successbox}

% ============================================================================
%                     PHẦN 10: KẾT LUẬN VÀ ĐỀ XUẤT
% ============================================================================
\newpage
\section{Kết luận và đề xuất}

\subsection{Tóm tắt dự án}

\begin{table}[H]
\centering
\caption{Tóm tắt thông tin dự án}
\renewcommand{\arraystretch}{1.5}
\begin{tabularx}{\textwidth}{|l|X|}
\hline
\rowcolor{primaryblue!10}
\textbf{Tiêu chí} & \textbf{Nội dung} \\
\hline
Tên dự án & CSV Translator Pro -- Phần mềm dịch thuật AI Offline \\
\hline
Mục tiêu & Dịch thuật đa ngôn ngữ offline, bảo mật tuyệt đối \\
\hline
Quy mô & 1000 người dùng \\
\hline
Tổng ngân sách & \textbf{789.800.000 VNĐ} \\
\hline
Thời gian triển khai & 6 tuần \\
\hline
Công nghệ & TranslateGemma-27B + FastAPI + React \\
\hline
GPU đề xuất & NVIDIA A100 80GB (Full Precision BF16) \\
\hline
Khả năng offline & 100\% sau khi triển khai \\
\hline
\end{tabularx}
\end{table}

\subsection{Đề xuất hội đồng}

\begin{dangerbox}[ĐỀ XUẤT PHÊ DUYỆT]
Kính đề xuất Hội đồng \textbf{phê duyệt kế hoạch thực hiện} dự án CSV Translator Pro với các nội dung:

\begin{enumerate}[label=\arabic*.]
    \item \textbf{Phê duyệt ngân sách}: 789.800.000 VNĐ (bao gồm dự phòng 10\%)
    \item \textbf{Phê duyệt cấu hình}: GPU NVIDIA A100 80GB + Server workstation
    \item \textbf{Phê duyệt timeline}: 6 tuần triển khai
    \item \textbf{Phê duyệt đội ngũ}: 8--10 nhân sự triển khai
    \item \textbf{Chỉ đạo phối hợp}: Các phòng ban hỗ trợ nhân sự, mạng, điện...
\end{enumerate}
\end{dangerbox}

\subsection{Lợi ích khi triển khai}

\begin{itemize}[leftmargin=1.5cm]
    \item \textbf{Tiết kiệm thời gian}: Dịch tự động hàng nghìn dòng CSV trong vài phút thay vì hàng ngày thủ công
    \item \textbf{Bảo mật tuyệt đối}: Dữ liệu không bao giờ rời khỏi mạng nội bộ
    \item \textbf{Chất lượng cao}: Mô hình AI 27 tỷ tham số của Google
    \item \textbf{Chi phí 1 lần}: Không phí hàng tháng như dịch vụ online
    \item \textbf{Hoạt động 24/7}: Không phụ thuộc Internet hay dịch vụ bên ngoài
    \item \textbf{Mở rộng}: Có thể nâng cấp phục vụ nhiều người dùng hơn
\end{itemize}

\vspace{1cm}
\begin{center}
\rule{10cm}{0.5pt}\\[0.5cm]
\textbf{Kính trình Hội đồng xem xét và phê duyệt}\\[1cm]
Ngày lập: \today
\end{center}

\end{document}
